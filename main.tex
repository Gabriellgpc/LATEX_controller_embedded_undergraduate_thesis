\documentclass[a4paper, 12pt]{article}

\usepackage[utf8]{inputenc}
\usepackage[portuguese,brazil]{babel}
\usepackage{amsmath, amssymb, amsfonts}
\usepackage[a4paper,left=3cm,right=2cm,top=3cm,bottom=2cm]{geometry}
% Caso não queira o 'geometry', pode-se usar:
% \usepackage{a4wide}
\usepackage{tabularx}
\usepackage{graphicx}
\usepackage{pslatex}
\usepackage{indentfirst}
\usepackage{setspace}
\usepackage{verbatim}
\usepackage{hyperref}
\usepackage{float}
\usepackage{algpseudocode,algorithm}
\usepackage[table,xcdraw]{xcolor}

\begin{document}

\thispagestyle{empty}

\begin{minipage}{0.15\linewidth}
\begin{flushleft}
\includegraphics[width=\linewidth]{UFRN}
\end{flushleft}
\end{minipage}\hfill
\begin{minipage}{0.65\linewidth}
\centering Universidade Federal do Rio Grande do Norte\\
Centro de Tecnologia\\
Coordenação de Engenharia de Computação
\end{minipage}\hfill
\begin{minipage}{0.15\linewidth}
\begin{flushright}
\includegraphics[width=0.6\linewidth]{EngComp}
\end{flushright}
\end{minipage}


\vfill

\begin{center}
% \LARGE{\bf Controle embarcado do tipo feedforward/backward para robô com 
% acionamento diferencial, usando encoders magnéticos de baixa resolução}

\Large{Controle embarcado do tipo feedforward/backward para robô com 
acionamento diferencial, usando encoders magnéticos de baixa resolução}
\end{center}

\normalsize
\vfill

\begin{flushleft}
{\bf Aluno(a):} Luís Gabriel Pereira Condados \\
{\bf Orientador acadêmico:}Adelardo Adelino Dantas de Medeiros\\
\end{flushleft}

\vfill

\begin{center}
Natal, RN, \today.
\end{center}

\newpage

\pagenumbering{roman}

\begin{center}
{\bf \Large Agradecimentos}
\end{center}

xxxxx.

\newpage

\tableofcontents

\newpage

% Colocar somente se necessáro...
\listoffigures
\addcontentsline{toc}{section}{Lista de Figuras}

\newpage
% Colocar somente se necessáro...
\listoftables
\addcontentsline{toc}{section}{Lista de Tabelas}

\section{INTRODUÇÃO}

A Equipe Poti de Futebol de Robôs surgiu no Departamento de Engenharia de Computação e Automação da Universidade Federal do Rio Grande do Norte (DCA-UFRN), sendo participante na categoria IEEE Very Small Size Soccer.
\\
Nesse tipo de categoria três robôs de cada time são utilizados de forma cooperativa com o objetivo de fazer o gol. Cada robô possui uma função no jogo, como: o goleiro, o atacante e o defensor, de forma que dependendo da estratégia de jogo eles podem se alternarem para melhor desempenho do time. A arquitetura do sistema do Futebol de Robôs do Time Poti é formada por vários módulos, sendo eles: visão, localização, estratégia, controle e transmissão.\\

O módulo de visão consiste de uma câmera que fotografa o campo com os jogadores e a bola, a taxa de captura da câmera é o fator que limita o tempo de amostragem de todo o sistema, atualmente a equipe utiliza uma taxa de amostragem de $100$ quadros/\emph{frames} por segundo(FPS). O módulo de localização fornece a posição da bola e dos jogadores através das cores que foram predefinidas. O módulo de estratégia é responsável pelo próximo passe que os robôs deverão executar, a partir das imagens atuais. O módulo de controle é responsável por converte as referências de posição e orientação, definidas pela estratégia, em sinais que correspondem as velocidades que devem ser aplicadas nas rodas direita e esquerda de cada robô e o módulo de transmissão envia esse sinal para os robôs.\\

O objetivo deste trabalho foi projetar e implementar um novo sistema de \textit{Hardware} e \textit{Firmware} para esses robôs. Os robôs antes deste trabalho possuíam um controlador da \emph{Atmega} de 32\emph{bits single core} usado em um módulo de desenvolvimento rápido (\emph{Arduino Nano}), um módulo \\emph{Bluetooth}, \emph{Drivers} motores e não possuía controle embarcado, ou seja, os sinais de controle dos motores direito e esquerdo direto eram provenientes diretamente do módulo de transmissão.\\

O novo \textit{Hardware}  conta com um microcontrolador dual core de 32\emph{bits} (ESP32) e \textit{Encoders} magnéticos de baixa resolução. Foi embarcado  nesse microcontrolador um controle de velocidade angular do tipo  \textit{FeedForward}/\textit{Backward}(PID), que deve atuar no par de motores dos robôs de  forma a controlar suas velocidades bem como reduzir as assimetrias entre  os motores direito e esquerdo.\\

Usando a interface \textit{Bluetooth} presente no microcontrolador, foi criado um  protocolo simples de comunicação para realizar operações de telemetria e telecomando nos robôs. O \textit{Firmware} opera utilizando os dois núcleos  executando tarefas distintas: um núcleo é responsável pela comunicação  com o servidor, por meio do \textit{Bluetooth} e do protocolo criado, bem como  por tratar as interrupções geradas pelos \textit{Encoders} rotativos magnéticos;  já o outro núcleo executa a rotina de controle, com um período de amostragem de pelo menos $5$ms, para operar em uma frequência de pelo menos duas vezes mais rápido que o sistema que está rodando no \emph{host}, que como já mencionado trabalha até à $100$fps ($10$ms).\\

% resolução do encoder: 3 pulsos por revolução ou
% 6 pulsos se contar os dois canais ou
% 12 bordas de subida e descida nos dois canais por revolução
Para medir a velocidade de rotação, foram utilizados \textit{Encoders} magnéticos  rotativos, com resolução de $3$ pulsos por canal por revolução do eixo do motor. Para diminuir a  incerteza na estimativa da velocidade, é aplicado um filtro de \textit{Kalman},  considerando a planta (motores) como um sistema de primeira ordem. Os  parâmetros da planta (constante de tempo, zona morta e ganho PWM x  Velocidade angular) são estimados/calculados por mínimos quadrados na  rotina de calibração presente no \textit{Firmware}, sendo em seguida utilizados  para determinar os parâmetros do controlador.

% Organização do trabalho
% Este trabalhado é organizado da seguinte maneira: O Capítulo 1 corresponde à introdução do trabalho, por meio da qual é feita uma contextualização do problema e a proposta de solução para o mesmo; o capítulo 2 traz a fundamentação teórica utilizada para o desenvolvimento deste trabalho; no capítulo 3, os objetivos gerais e específicos são abordados; no capítulo 4 são apresentados os materiais e métodos utilizados para implementação do sistema; o capítulo 5 contém os resultados e discussões em que podem ser observadas as telas do sistema proposto; no capítulo 6 as conclusões deste trabalho são levadas em consideração, assim como sugestões de trabalhos futuros.
\section{REFERENCIAL TEÓRICO}
Nesta seção será apresentado os principais conceitos utilizados neste trabalho, como a modelagem do motor de corrente contínua utilizada; Filtro de Kalman; Teórica de Controle para sistema de primeira ordem; Controlador FeedForward; Comunicação Bluetooth entre outras coisas.

\subsection{Modelagem Motor de Corrente Contínua}
% Colocar imagem
Motores geram potência mecânica, de forma que a velocidade de rotação $\omega$ é o sinal de saída e a tensão aplicada é o sinal de entrada. 

% figura retirada do livro texto da disciplina de modelagem.
% FAZER FIGURA PRÓPRIA
\begin{figure}[H]
    \centering
    \includegraphics{imagens/ilustracoes/motor_cc_diagrama_modelo.png}
    \caption{Diagrama esquemático de um motor CC}
    \label{fig:modelo_motorcc}
\end{figure}

A figura \ref{fig:modelo_motorcc} apresenta o diagrama esquemático para um motor de corrente contínua (CC) controlado pela armadura, ou seja, o sinal de entrada é a tensão aplicada na armadura ($v_a$). Nesse diagrama a carga está sendo modelada por um momento de inércia $J$ e um atrito viscoso de coeficiente $b$.

\begin{align*}
    v_g &= K_{1}\phi\omega= K_{1}K_{2}i_{f}\omega = K_{m}\omega\\
    \tau &= K_{1} \phi i_{a}= K_{1} K_{2}i_{f} i_{a} = K_{m}i_{a}
\end{align*}

A constante $K_{m}$ é conhecida como a constante do motor. Devido a relação apresentada anterior, possível modelar o circuito equivalente do motor CC como na imagem \ref{fig:eq_eletrico_motorcc}.
% figura retirada do livro texto da disciplina de modelagem.
% FAZER FIGURA PRÓPRIA
\begin{figure}[H]
    \centering
    \includegraphics{imagens/ilustracoes/motor_cc_eq_eletrico.png}
    \caption{Equivalente elétrico de um motor CC}
    \label{fig:eq_eletrico_motorcc}
\end{figure}

Do circuito da figura \ref{fig:eq_eletrico_motorcc} extrai-se a seguinte função de transferência:

\begin{equation*}
    \frac{\Omega(s)}{V_a{s}} = \frac{K_m}{JL_{a}s^2 + \left(JR_a + BL_a \right)s + BR_a + K_{m}^2} \left[\frac{ rad.s^{-1}}{V}  \right]
\end{equation*}

Caso a impedância da armadura seja desprezada $(L_a \xrightarrow{} 0)$:

\begin{equation}
    \frac{\Omega(s)}{V_{a}(s)} = \frac{K_m}{JR_{a}s + BR_{a} + K_{m}^2} = \frac{K}{T_{m}s + 1} \left[\frac{ rad.s^{-1}}{V}  \right]
    \label{eq:motor_transf_func}
\end{equation}

Portando, caso a impedância da armadura seja desprezada, a função de transferência do motor que relacionada a velocidade angular com a tensão de entrada se comporta como um sistema de primeira ordem.

\subsection{Sistemas de Controle}

\begin{figure}[H]
    \centering
    \includegraphics[width=\textwidth]{imagens/ilustracoes/sistema_de_controle_completo.eps}
    \caption{Diagrama de um sistema de controle \textit{FeedForward} + \textit{Backward}}
    \label{fig:diagrama_sistema_de_controle_feedforward_backward}
\end{figure}

\subsubsection{Controlador PID}

\subsubsection{Controlador FeedForward}

\subsubsection{Controlador FeedForward/Backward}

% Breve introdução à teoria de controle 
% modelagem utilizada para o trabalho atual, sistema motor/roda considerado
\subsection{Mínimos Quadrados}

\subsection{Filtro de Kalman}
% Introdução ao filtro de Kalman

Modelo do sistema:
\begin{equation}
\textbf{x}_k = F_x x_{k-1} + B_k u_k + w_k; w_k \sim N(0, Q_k)
\end{equation}

Modelo da medição:
\begin{equation}
z_k = H_k x_k + v_k; v_k \sim N(0, R_k)
\end{equation}

Predição:
\begin{align*}
    \check{x}_k &= F_k \hat{x}_{k-1} + B_k u_k\\
    \check{P}_k &= F_k \hat{P}_{k-1} F^T_k + Q_k
\end{align*}

Atualização:
\begin{align*}
    K_k &= \check{P}_k H^T \left( H_k \check{P}_k H^T_k + R_k\right)^{-1}\\
    \hat{x}_k &= \check{x}_k + K_k\left( z_k - H_k \check{x}_k \right)\\
    \hat{P}_k &= \left(I - KH_k \right)\check{P}_k
\end{align*}

% \subsubsection{Filtro de \textit{Kalman} para um sistema de primeira ordem}

\subsection{Modulação por Largura de Pulso}
% Resumo sobre PWM

\subsection{Bluetooth}
% Breve introdução à historia do bluetooth e utilidade
% destacar: custo energético, velocidade de operação, robustez a interferências

A  tecnologia \emph{Bluetooth}  suporta várias opções de topologia, incluindo conexões simples ponto a ponto. Operando na faixa de frequência industrial, científica e médica de $2,4$ GHz, a tecnologia \emph{Bluetooth} suporta várias opções de rádio.
    
O rádio \emph{Bluetooth} BR/EDR opera com baixo consumo de energia e também utiliza uma abordagem robusta de \textit{Adaptive Frequency Hopping}, transmitindo dados por $79$ canais. O \textit{Bluetooth} BR/EDR inclui várias opções de \textit{PHY} que suportam taxas de dados de $1$ Mb a $3$ Mb e suporta vários níveis de energia, de $1$mW a $100$ mW, além de várias opções de segurança.

% aprofundar mais? 

\subsection{\textit{Encoders} Rotativos}
% Referencias:
% https://www.roboticsbusinessreview.com/news/differences-between-encoder-resolution-accuracy-and-precision/

% FALAR SOBRE O FUNCIONAMENTO BASICO
% CITAR OS DIFERENTES TIPOS
% FALAR DO GRAY CODE
% FALAR DO ERRO DE QUANTIZAÇÃO

\subsection{FreeRTOS}
% É um sistema operacional de tempo real.
% introduzir o que um RTOS, e introduzir o FreeRTOS
O \emph{FreeRTOS} é um sistema operacional de tempo real para microcontroladores, e é o utilizado no microcontrolador usado neste trabalho. Ele é desenvolvido por diversas empresas e é um dos mais populares e usados no mercado. Com ele é possível gerenciar os recursos do microcontrolador, como lançar vários processos em mais de um núcleo, dentre outras coisas, assim como um sistema operacional convencional, porém mais leve, pois possui menos funcionalidades. O \emph{FreeRTOS} é distribuído gratuitamente sobre a licença \textit{MIT}.

% FALTA COMPLEMENTAR

\section{MATERIAIS E MÉTODOS}
Como já mencionado, o objetivo básico do projeto foi projetar os robôs da equipe de futebol de robôs do laboratório de robótica do DCA, sendo assim os robôs devem ter suas dimensões limitadas a um cubo de $75$mm de aresta, essa foi umas das principais restrições levada em conta para o projeto eletrônico dos robôs, além do fator econômico.  A parte estrutural do robô não será abordada neste trabalho, mas sim a parte eletrônica e de software.

\subsection{Componentes}
% Listas os componentes utilizados na montagem e mostrar algumas de suas
% principais características e sua finalidade no projeto
% mostrar o diagrama eletrônico
% Falar um pouco sobre o funcionamento dos sensores
% Falar como foi implementado a leitura, as otimizações feitas
São quatro o número de componentes básicos que compõem os robôs presentes neste trabalho, sendo eles: um par de atuadores (motor direito e esquerdo), um par de sensores de rotação (\textit{Encoder}s magnéticos), um driver motor multicanal, um microcontrolador e bateria recarregável. Devido as características dos componentes escolhidos para o projeto, apenas estes quatro tipos foram suficiente para compor a eletrônica do robô de forma a respeitar as restrições dimensionais, realizar o controle feedforward/backward de forma eficiência e com um bom período de amostragem e baixo gasto energético, além de um baixo custo financeiro.

A seguir serão apresentados mais detalhes dos componentes supracitados.

% 30:1 Micro Metal Gearmotor HP 6V with Extended Motor Shaft
\begin{figure}[H]
    \centering
    \includegraphics[width=3cm]{imagens/eletronica/motor_com_encoder.jpg}
    \caption{Micro Motor de 6V com caixa de redução de 30:1 e \textit{Encoder} magnético.}
    \label{fig:motor_com_encoder}
\end{figure}

A figura \ref{fig:motor_com_encoder} mostra o motor escolhido já com o \textit{Encoder} magnético colocado em seu eixo estendido (placa de circuito impresso com um Imã natural em forma de disco), esse é um micro motor de $6$V com uma caixa de redução de $\approx 30:1$ da \textit{Pololu}[?].

\begin{figure}[H]
    \centering
    \includegraphics[width=5cm]{imagens/eletronica/encoder_frente_verso.jpg}
    \caption{Par de Encoders Magnéticos de $12$ pulsos por revolução ($12$CPR)}
    \label{fig:encoder}
\end{figure}

\begin{figure}[H]
    \centering
    \includegraphics[width=3cm]{imagens/eletronica/driver.jpg}
    \caption{\textit{Driver} Motor TB6612FNG.}
    \label{fig:driver_motor}
\end{figure}

\begin{figure}[H]
    \centering
    \includegraphics[width=3cm]{imagens/eletronica/esp32_kit.png}
    \caption{Placa de desenvolvimento ESP32 Dev1.}
    \label{fig:esp32_kit}
\end{figure}

\begin{figure}[H]
    \centering
    \includegraphics[width=5cm]{imagens/robo_completo_explodido.png}
    \caption{Vista explodida do robô completo.}
    \label{fig:robo_completo_explodido}
\end{figure}

\subsection{Placa de Circuito Impresso}
% CIRCUITO ELÉTRICO/ELETRÔNICO COMPLETO
% MOSTRAR/EXPLICAR: PLACA DE CIRCUITO IMPRESSO DESENVOLVIDA
% TODO:
% ref:  https://produza.ind.br/gestao/pre-requisitos-tecnicos-para-montar-um-projeto-eletronico/

% \begin{figure}[H]
%     \centering
%     \includegraphics[width=\textwidth]{imagens/eletronica/placa/placa_controle_completa.pdf}
%     \caption{Caption}
%     % \label{fig:my_label}
% \end{figure}

% \begin{figure}[H]
%     \centering
%     \includegraphics[width=\textwidth]{imagens/eletronica/placa/placa_driver_completa.pdf}
%     \caption{Caption}
%     % \label{fig:my_label}
% \end{figure}

\begin{enumerate}
    \item \textbf{Concepção inicial}:
        O principal objetivo por trás de se fazer uma nova placa de circuito eletrônico é acomodar os componentes, respeitando os limites das dimensões estabelecidos pela competição na qual os robôs serão utilizados (caber dentro de um cubo de $75$ mm de aresta). A placa deve conter o microcontrolador, no seu kit de desenvolvimento, \textit{Driver motor} para acionamento dos motores DCs, os \textit{Encoders} e ser alimentada por duas baterias de $1$ célula do tipo \textit{Lipo}.
        
    \item \textbf{Elaboração dos esquemáticos eletrônicos}:
        O esquemático foi a parte mais simples, pois não houve grandes mudanças nessa parte, com relação aos projetos de anos anteriores. A maior mudança foi o microcontrolador, que provocou dificuldades maiores na etapa seguinte, a elaboração do \textit{Layout}, devido às dimensões dos componentes.
        
        % inserir imagem do esquemático geral aqui
        
        \begin{figure}[H]
            \centering
            \includegraphics[width=\textwidth]{imagens/eletronica/placa/esquematico_completo.pdf}
            \caption{Esquemático.}
        \end{figure}
        
    \item \textbf{Elaboração do layout}:
        % inserir imagem do layout final aqui
        Este ponto foi o mais crítico nessa tarefa, devido à restrição de tamanho da placa ser de um quadrado com até $55$mm de lado. A solução adotada foi fazer em duas camadas, ou seja, duas placas cobreadas, ambas dupla face, dividindo os componentes. Em uma placa foi comportado o \textit{Driver}, bem como os conectores para os motores com os sensores e os conectores das baterias, e na outra apenas o microcontrolador. Para a conexão entre as placas foi utilizado um conector do tipo \textit{Head}, um macho e uma fêmea. Dessa forma, as placas conseguiram respeitar o limite dimensional e comportar todos os componentes necessários.
    \item \textbf{Realização de testes}:
        % faltou imagem de testes aqui
        Foram realizados testes antes da concepção da PCI, em \textit{Protoboards} para conferir se o circuito está funcionando como esperado e após a confecção, para se verificar a qualidade da confecção das placas.
        
        Também foram realizados testes individuais nos componentes, principalmente nos sensores, com o uso de osciloscópios. Verificou-se o funcionamento correto dos \textit{Encoders} e também conferiu-se se a distância entre os sensores poderia estar gerando interferência um no outro. Os resultados desses testes foram que todos os \textit{encoders} utilizados estão em bom estado, ou seja, funcionando como esperado e a distância que eles ficarão ao serem acomodados na estrutura não causa interferência um no outro.
    \item \textbf{Verificação e validação}
        Após testes individuais, de cada componente, foram realizados testes com a montagem completa, ou seja, os robôs montados por completo com as PCIs, baterias, motores e sensores. A validação foi por meio de controle manual dos robôs e testes simples de leitura de \textit{Encoder}, pois nessa etapa o \textit{Firmware} ainda não havida sido implementado.
\end{enumerate}

\subsection{O \emph{Firmware}}
% Falar sobre como foi feito a divisão das atividades no microcontrolador
% Mostrar ilustração dessa divisão, para ter-se uma visão global das rotinas e suas relações de dependência

\subsubsection{Rotina de Calibração}

Rotina responsável por realizar a identificação dos parâmetros do modelo dos motores direito e esquerdo do robô: constante de tempo; ganho de malha aberta; parâmetros do controlador \emph{FeedForward}; velocidade máxima de cada motor. Bem como calcular os ganhos para o controlador \emph{PID} de forma a se obter uma resposta pré-definida em malha fechada.\\

A rotina de calibração consiste em três etapas que são repetidas para todas as configurações: motor direito para frente; motor direito para trás; motor esquerdo para frente e motor esquerdo para trás. Cada etapa é descrita a seguir: 

% TO DO:
% ORGANIZAR E REPENSAR A FORMA DE EXIBIR AS ETAPAS
% FIGURAS ILUSTRANDO O COMPORTAMENTO GERAL DAS FUNÇÕES QUE ESTÃO SENDO USADAS NA INTERPOLAÇÃO
% PSEUDO CÓDIGOS TALVEZ
\begin{itemize}
    \item \textbf{ETAPA 1}: Estimar a zona morta e o ganho da planta em malha aberta\\
    
    Para isso é realizado a coleta de $N$ pontos ($\omega$,$u$), o primeiro ponto é coletado para $u = \pm1$(valor máximo no sentido de giro atual) e é realizado sucessivos decréscimos neste valor até a parada do motor ($\omega = 0$). \\
    
    Aplicar sinal de controle atual ($u_i$);\\
    Aguardar um tempo fixo, pré-determinado, para garantir a leitura de $\omega_{ss}$(velocidade de máxima/velocidade de regime)
    Armazenar ($u_i$,$\omega_{ss}$).\\
    
    A aquisição destes pontos ocorre da maior velocidade para a menor devido à zona morta ser mais baixa neste sentido, por causa da inercia do motor.\\
    
    Ao se encerrar a coleta destes pontos ($\omega_{ss} = 0$ detectado) é realizado uma regressão linear por \textit{MMQ}, tendo $\omega$ no eixo das ordenadas e $u$ no eixo das abscisas, o coeficiente angular dessa reta relaciona a velocidade angular ($\omega$) com o sinal de controle ($u$), já o coeficiente linear representa a zona morta, ou seja, o menor $u$ que pode iniciar o giro do motor.
    
    \begin{equation*}
        u(\omega) = a\omega + b
    \end{equation*}
    
    Com esses parâmetros dessa reta é possível estimar o ganho de malha aberta da planta da seguinte forma:
    
    \begin{equation*}
        K = \frac{(u_{max} - b)}{a}
    \end{equation*}
        
    \begin{figure}[H]
        \centering
        \includegraphics[width=0.5\textwidth]{imagens/ilustracoes/omega_x_sinal_controle.eps}
        \caption{Comportamento da curva $u(\omega)$.}
        \label{fig:ilustracao_omega_x_pwm}
    \end{figure}    
        
    \item \textbf{ETAPA 2}: Estimar a constante de tempo
    
    Para obter a constante de tempo da planta na configuração atual, faz-se uso mais uma vez de interpolação por \textit{MMQ} e tira-se proveito do conhecimento do ganho da planta para simplificar e tornar possível essa interpolação de forma simples. Para estimar a constante de tempo obtém-se $M$ pontos ($t$,$\omega$) e aplica-se o \textit{MMQ} para uma interpolação linear, para isso é necessário fazer a seguinte alteração:
        

    \begin{align*}
        \omega(t) &= K\left( 1 - e^{-t/T_m} \right)\\
        \ln{\omega(t)} &= \ln\left[K( 1 - e^{-t/T_m})\right]\\
        \ln\left(1 - \frac{\omega(t)}{K} \right) &= -\frac{t}{T_m}\\
        y_{aux}(t) &= -\frac{t}{T_m}
    \end{align*}
    
    Convertendo $\omega(t)$ para $y_{aux}(t)$ o coeficiente angular resultante da interpolação linear será: $coef.angular = -\frac{1}{T_m}$, dessa forma obtemos a constante de tempo.
    
    \item \textbf{ETAPA 3}: Calcular os parâmetros do controlador \textit{PID}
    
    Como apresentado na seção de referencial teórico é possível relacionar o ganho do controlador proporcional com o polo desejado para o sistema em malha fechada, por meio da relação (??). Esse cálculo só é possível devido a identificação dos parâmetros da planta resultante das etapas anteriores. O polo desejado para todas as configurações da planta é uma constante pré-definida pelo usuário.
    
\end{itemize}

Ao se passar por todas as configurações de motor/sentido a rotina seleciona a menor velocidade máxima apresentada por alguma dessas configurações e armazena esta velocidade como sendo a velocidade máxima atingida pelos motores deste robô, isso é importante, para assegurar que a referencia $\omega_{ref}$ seja factível para todas as configurações. \\

Por fim os resultados são armazenados na memória permanente do microcontrolador, sendo atualizado/sobrescrita apenas ao final da próxima chamada da rotina de calibração.

\subsubsection{Rotina de Comunicação}
% TODO:
% IDEIA: ILUSTRAR A INTERAÇÃO ENTRE A ROTINA PRINCIPAL E A INTERRUPÇÃO
A rotina de comunicação opera em um \emph{loop} infinito no núcleo principal do microcontrolador e é responsável por tratar os telecomandos recebidos pelo \textit{Bluetooth}. Ao ser identificado um recebimento de mensagem pela sinal de interrupção da comunicação \textit{Bluetooth} é acionado a função de tratamento de interrupção correspondente que possui como única função encaminhar as mensagens válida (verificar cabeçalho) para a rotina principal de comunicação. Isso é feito para evitar sobrecarregar a interrupção, já que ela deve operar em altas frequências.\\

Na rotina principal a mensagem é interpretada e caso ela seja identificado um telecomando válido, será executado a devida resposta, conforme apresentado a seguir.

O protocolo implementado, foi pensado para conter até três grandes campos, o \textbf{\textit{Head}} com 4 bits de preambulo, para ajudar a sincronizar os pacotes, o \textbf{\textit{Cmd}} também com 4 bits, possibilitando assim até 16 comandos distintos e o campos de argumentos com tamanho variável. Foram implementados 7 comandos. 

\input{tabelas/defines_4protocol.tex}
    
\textbf{Comandos}
\begin{itemize}
    \item \textbf{CMD\_REQ\_CAL}:\\
        \textit{Host} envia, para solicitar os dados provenientes da calibração do controlador \textit{feedforward}. O escravo (robô) envia 4 \emph{floats}, referente aos coeficientes do controlador.
    \item \textbf{CMD\_REQ\_OMEGA}:\\
        \textit{Host} envia, para solicitar as velocidades atuais de ambos os motores, em $rad/s$. O escravo responde com dois \emph{floats}, referentes aos ômegas em cada motor.
    \item \textbf{CMD\_CALIBRATION}:\\
        \textit{Host} envia, para fazer com que o robô inicie sua rotina de calibração do controlador.
    \item \textbf{CMD\_IDENTIFY}:\\
        \textit{Host} envia, fazendo com que o robô inicia sua rotina de identificação. O \textit{Host} deve enviar o \emph{bitstream} da seguinte forma:\\
        
        \input{tabelas/cmd_identify.tex}
        
        Sendo o campos \textbf{options} de 1 byte, contendo a informação de qual motor será feita a identificação e se deve ser usado o controlador.
        
        Ao concluir a rotina de identificação, o robô responde enviando o vetor de ômegas medidos, durante a rotina, para o \textit{host}, que deve estar aguardando recebê-las. A quantidade de dados será $\frac{timeout}{steptime}*4$ bytes, portando o \textit{host} deve estar aguardando exatamente essa quantidade de bytes.
        
        
    \item \textbf{CMD\_SET\_POINT}:\\
        
        \input{tabelas/cmd_setpoint.tex}
        
        Neste os campos de \textbf{sense\_x} indicam o sentido de rotação do motor, 0 para trás e 1 para rodar para frente (convertidos em sinal dos ômegas de setpoint), portando só ocupam 1 bit, já os campos referentes aos ômegas desejados ocupam 15 bits, sendo assim é possível enviar referências com uma precisão de $1.0/2^{15}$, já que as referências serão enviadas inteiras  (0 - $2^{15}$) e mapeadas de $-1.0$ a $1.0$, indicando uma porcentagem da referência da velocidade máxima do robô. Ou seja os campos referentes aos \textit{setpoints} contêm a porcentagem da velocidade máxima do robô.
        
    \item \textbf{CMD\_CONTROL\_SIGNAL}:\\
        
        O comando difere apenas o campo de \textbf{cmd} do comando anterior. O restante da estrutura é exatamente igual, pois a principal diferença ocorre no microcontrolador. Em vez dos campos referentes aos ômegas serem porcentagens da velocidade máxima que será convertido em \textit{Setpoint} para o controlador, neste comando o robô irá interpretar esses campos como sendo sinais de controle (após convertê-los para \emph{float} de $-1.0$ a $1.0$).
        
    \item \textbf{CMD\_PING}:\\
        Neste comando o \textit{host} pode enviar qualquer mensagem no campo de argumentos, pois o robô irá apenas responder com a mesma mensagem. Este comando é útil para testar conexão e testar a latência da conexão.
    
\end{itemize}

% TABELA TEMPORÁRIA
\include{tabelas/tabela_comandos}

\subsubsection{Rotina de Controle}
% TODO:
O controle é executado sozinho no núcleo secundário do ESP32 em \emph{loop} infinito com uma taxa de atualização de $5ms$. A rotina consiste em receber as velocidades providas pela de leitura dos sensores por meio de mensagens entre processos, em seguida é realizado a etapa de controle propriamente dita, que consiste em aplicar a ação do controle \textit{FeedForward} somado a ação do controle de malha fechada, tendo como entrada a última referência de percentual de velocidade lida pela rotina de comunicação multiplicada pela velocidade máxima considerada para a rotação do eixo do motor, para passar a referência para $rad/s$. \\

Ao final do fluxo de controle tem-se o sinal de controle, este por sua vez é utilizado para configurar o sentido de giro e o sinal \emph{PWM} que deve ser aplicado no \emph{Driver} motor correspondente.

\subsubsection{Rotina de Leitura dos Sensores}
% TODO:
% REALIZAR ANÁLISE DE FREQ. MÁXIMA DE OPERAÇÃO DOS ENCODERS VS BANDA DISPONIVEL NAS INTERRUPÇÕES RESPONSAVEIS PELA LEITURA DOS ENCODERS

Há duas interrupções associadas aos sinais provenientes dos \emph{Encoders} rotativos, uma para cada motor, cujo objetivo é calcular/medir a velocidade de rotação do eixo do motor ($\omega_{medido}$) bem como aplicar o filtro de \emph{Kalman} para uma melhor estimativa da mesma. As interrupções são provocadas pelas bordas dos pulsos de ambos os canais, fazendo com que a resolução do sensor seja utilizada ao máximo, pois dessa maneira os sensores que possuem uma resolução de três pulsos completos por revolução (em cada canal, ver Figura \ref{fig:ilustracao_uma_revolucao}) consegue acionar doze ($12$) vezes a interrupção que irá computar o $\omega_{motor}$, passando assim a ser ter uma resolução de doze pulsos por revolução.\\

Para o cálculo do módulo da velocidade de rotação do eixo do motor faz-se:
\begin{equation}
    |\omega_{medido}| = \frac{2\pi}{NPR}\frac{1}{\Delta{t}}
    \label{eq:modulo_omega_medido}
\end{equation}

Onde $NPR$ é o número pulsos por revolução em um canal. Neste trabalho o $NPR$ é igual a três($3$). Já o intervalo $\Delta{t}$ é o intervalo entre bordas iguais em um canal, a Figura \ref{fig:sinal_em_quadratura_delta_t} ilustra esse intervalo para as bordas de subida de ambos os canais.\\

\begin{figure}[H]
    \centering
    \includegraphics[width=0.5\textwidth]{imagens/ilustracoes/sinal_enquadratura_uma_revolucao.eps}
    \caption{Ilustração do sinal em quadratura em uma revolução completa do eixo do motor no sentido horário.}
    \label{fig:ilustracao_uma_revolucao}
\end{figure}

\begin{figure}[H]
    \centering
    \includegraphics[width=0.6\textwidth]{imagens/ilustracoes/sinal_em_quadratura_sentido_CCW_detalhada.eps}
    \caption{Ilustração de um sinal em quadratura para uma revolução do eixo do motor sentido anti-horário para um \emph{Encoder} com a resolução de três pulsos por revolução, com destaque para o intervalo de tempo entre as bordas de subida de um mesmo canal.}
    \label{fig:sinal_em_quadratura_delta_t}
\end{figure}

\input{tabelas/tabela_simple_code}

Já para se obter o sentido de rotação do motor, faz-se uso do padrão \emph{Gray Code} gerado pela diferença de fase entre os diferentes canais de um mesmo sensor, uma maneira de fazer isso é ler os \emph{GPIO}'s associados aos canais do \emph{Encoder} e verificar o padrão em binário e inferir o sentido de rotação, a Figura \ref{fig:cw_signal} ilustrado isso para uma rotação no sentido horário e a Figura \ref{fig:ccw_signal} o anti-horário, esses códigos são apresentados na Tabela \ref{tab:tabela_simple_code}. Porém esse procedimento apresentou ser pouco eficiente em medias e altas rotações, devido a alta taxa de erro na inferência do sentido. \\


A abordagem adotada aqui foi armazenar os dois \emph{bits}(sendo canal A bit mais significado) e concatenar/somar com os últimos dois \emph{bits} (da leitura anterior) deslocados em dois (equivalente à multiplicar por $2^2$ ou operar bit-a-bit: $bits_{anteriores} \ll 2$), criando assim um padrão com $4$ \emph{bits}, sendo os dois mais significados o padrão da leitura anterior e os dois menos significativos a leitura atual. Esse procedimento é ilustrado para uma rotação no sentido horário e no sentido anti-horário respectivamente nas Tabelas \ref{tab:tabela_gray_code_cw} e \ref{tab:tabela_gray_code_ccw}, dessa forma gera-se quatro($4$) padrões/códigos que caracterizam um tipo de rotação. Esse padrão de $4$\emph{bits} é armazenado de forma estática em um vetor(uma tabela de cola/ \emph{lookup table}) nas rotinas de ambos os motores, esse vetor mapeia o código binário em $1$, $-1$ ou zero para as combinações que não caracterizam um sentido de giro, o sinal do valor corresponde ao sentido horário ou anti-horário e depende do motor, um exemplo de \emph{lookup table} é apresentado na Tabela \ref{tab:lookup_table}.

\begin{figure}[H]
    \centering
    \includegraphics[width=0.7\textwidth]{imagens/ilustracoes/sinal_enquadratura_sentido_CW.eps}
    \caption{Sinal em quadratura para rotação no sentido horário.}
    \label{fig:cw_signal}
\end{figure}

\input{tabelas/tabela_cw_gray_code}

\begin{figure}[H]
    \centering
    \includegraphics[width=0.7\textwidth]{imagens/ilustracoes/sinal_enquadratura_sentido_CCW.eps}
    \caption{Sinal em quadratura para rotação no sentido anti-horário.}
    \label{fig:ccw_signal}
\end{figure}

\input{tabelas/tabela_ccw_gray_code}

% Please add the following required packages to your document preamble:
% \usepackage{graphicx}
\begin{table}[H]
\centering
\resizebox{0.8\textwidth}{!}{%
\begin{tabular}{c|ccccllllllllllll}
\textbf{Índice} & 0 & 1  & 2 & 3 & 4 & 5 & 6 & 7  & 8  & 9 & 10 & 11 & 12 & 13 & 14 & 15 \\ \hline
\textbf{Valor}  & 0 & -1 & 1 & 0 & 1 & 0 & 0 & -1 & -1 & 0 & 0  & 1  & 0  & 1  & -1 & 0 
\end{tabular}%
}
\caption{Lookup table.}
\label{tab:lookup_table}
\end{table}

Com a \emph{lookup table} e o módulo da velocidade é possível calcular a velocidade de rotação da seguinte forma:

\begin{equation}
    \omega_{medido} = \frac{2\pi}{NPR}\frac{table[code]}{\Delta{t}}
\end{equation}

A próxima etapa é a filtragem dessa velocidade, a rotina aplica o filtro de \emph{Kalman} (mais detalhes na seção de referencial teórico) considerando $omega(t)$ um sistema de primeira ordem, como apresentado na seção sobre modelagem de motores de corrente contínua essa pode ser uma boa aproximação se desconsiderar a influência da indutância interna do motor, isso faz com as variáveis do filtro sejam modeladas para:


\begin{equation*}
\begin{cases}
    \textbf{x}_k = \left[ \omega_k \right]\\
    z_k = x_k = \omega_k\\
    F_k = 1\\
    H_k = 1
\end{cases}
\end{equation*}

Modelo da \textbf{medição}:
\begin{align*}
z_k = \omega_{medido}
\end{align*}

Com isso a etapa de \textbf{predição} do filtro torna-se:
% MUDAR O SIMBOLO QUE FAZ REFERENCIA À ENTRADA (u) DE ENTRADA
\begin{align*}
    \check{\omega}_k &= \hat{\omega}_{k-1} + u_k\left( 1 - e^{-\Delta{t}/T_m} \right)\\
    \check{P}_k &= \hat{P}_{k-1} + Q_k
\end{align*}

Sendo $u_k$ a entrada no instante $k$, $\Delta t = t_f - t_0$. $\Delta t$ é relativo ao sinal de entrada $u_k$, sendo $t_0$ o instante que o sinal é aplicado e $t_f$ o instante atual $k$.


E a etapa de atualização \textbf{Atualização} é:

\begin{align*}
K_k &= \check{P}_k \left( \check{P}_k + R_k \right)^{-1} = \frac{\check{P}_k}{\check{P}_k + R_k}\\
\hat{\omega}_k &= \check{\omega}_k + K_k \left( \omega_{k_{medido}} - \check{\omega}_k \right)\\
\hat{P}_k &= \left( 1 - K_k \right) \check{P}_k
\end{align*}


% colocar aqui pseudo código da rotina completa

% \begin{algorithm}
% \caption{A Hough Transform for Squares Detection}
% \label{alg:hough_for_squares}
% \begin{algorithmic}[1]
% \State initialize $M$ with zero. \Comment{Voting matrix initialized with zero.}
% \ForAll{$(x',y')$ in $f(x,y)$}
%     \If{$|\nabla{f(x',y')}| \geq $ edge threshold } \Comment{It's an edge.}
        
%         \State $p_0 \gets (x',y')$
%         \State $\theta \gets \angle{\nabla{f(x',y')}} \mod{90^\circ}$ \Comment{Square's angle.}
%         \State $\Vec{N_0} \gets \nabla{f(x',y')}/|\nabla{f(x',y')}|$
%         \State $\Vec{N_1} \gets -\nabla{f(x',y')}/|\nabla{f(x',y')}|$
%         \State $\Vec{N_2} \gets$ rotate $\Vec{N_1}$ on $90^\circ$\\
        
        
%         \State Search for an edge point going in the direction of $\Vec{N_1}$ from $p_0$. \Comment{That will be the $p_1$ point.}\\
        
%         \State $l \gets |p_0 - p_1|$ \Comment{Square's size.}
        
%         \State $p_{middle} \gets (p_1 + p_0)/2$\\
        
%         \State Search for an edge point going in the direction of $\Vec{N_2}$ from $p_{middle}$. \Comment{That will be the $p_2$ point.}\\
        
%         \State $p_{center} \gets p_2 - \Vec{N_2}*l/2$ \Comment{$p_{center} = (x_c,y_c)$}
        
%         \State $M[x_c][y_c][l][\theta] \gets M[x_c][y_c][l][\theta] + 1$
        
%     \EndIf
% \EndFor
% \end{algorithmic}
% \end{algorithm}


% \subsubsection{Rotina de Telemetria}
% TODO:
% NOTA:
% Não sei se seria interessante falar dessa. Ao ser ativada ela inicia o armazenamento de vários dados, como velocidades, setpoint, velocidade filtrada, velocidade não filtrada, variáveis do filtro... grava esses dados durante x segundos e ao final envia tudo ao host via bluetooth
\input{tex/resultados}
\section{CONCLUSÃO}

Lorem ipsum dolor sit amet, consectetur adipiscing elit. Fusce malesuada posuere viverra. In dignissim lacus at arcu tincidunt sollicitudin. Donec sed metus eget dui tincidunt faucibus. Vivamus eu turpis risus. Pellentesque sit amet consequat nibh. Fusce dignissim tortor suscipit ex elementum feugiat. Nullam blandit urna fermentum ante gravida, sed aliquam magna congue. Vivamus turpis eros, eleifend sed faucibus eget, ultrices ut ligula. In in iaculis massa. Orci varius natoque penatibus et magnis dis parturient montes, nascetur ridiculus mus. Suspendisse pretium consequat tristique. Vestibulum diam ligula, bibendum ut quam nec, hendrerit lobortis eros. Aliquam quis tellus eget felis laoreet tempor.

\bibliography{bibliografia}
\end{document}


