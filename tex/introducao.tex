\section{INTRODUÇÃO}

A Equipe Poti de Futebol de Robôs surgiu no Departamento de Engenharia de Computação e Automação da Universidade Federal do Rio Grande do Norte (DCA-UFRN), sendo participante na categoria IEEE Very Small Size Soccer.
\\
Nesse tipo de categoria três robôs de cada time são utilizados de forma cooperativa com o objetivo de fazer o gol. Cada robô possui uma função no jogo, como: o goleiro, o atacante e o defensor, de forma que dependendo da estratégia de jogo eles podem se alternarem para melhor desempenho do time. A arquitetura do sistema do Futebol de Robôs do Time Poti é formada por vários módulos, sendo eles: visão, localização, estratégia, controle e transmissão.\\

O módulo de visão consiste de uma câmera que fotografa o campo com os jogadores e a bola, a taxa de captura da câmera é o fator que limita o tempo de amostragem de todo o sistema, atualmente a equipe utiliza uma taxa de amostragem de $100$ quadros/\emph{frames} por segundo(FPS). O módulo de localização fornece a posição da bola e dos jogadores através das cores que foram predefinidas. O módulo de estratégia é responsável pelo próximo passe que os robôs deverão executar, a partir das imagens atuais. O módulo de controle é responsável por converte as referências de posição e orientação, definidas pela estratégia, em sinais que correspondem as velocidades que devem ser aplicadas nas rodas direita e esquerda de cada robô e o módulo de transmissão envia esse sinal para os robôs.\\

O objetivo deste trabalho foi projetar e implementar um novo sistema de \textit{Hardware} e \textit{Firmware} para esses robôs. Os robôs antes deste trabalho possuíam um controlador da \emph{Atmega} de 32\emph{bits single core} usado em um módulo de desenvolvimento rápido (\emph{Arduino Nano}), um módulo \\emph{Bluetooth}, \emph{Drivers} motores e não possuía controle embarcado, ou seja, os sinais de controle dos motores direito e esquerdo direto eram provenientes diretamente do módulo de transmissão.\\

O novo \textit{Hardware}  conta com um microcontrolador dual core de 32\emph{bits} (ESP32) e \textit{Encoders} magnéticos de baixa resolução. Foi embarcado  nesse microcontrolador um controle de velocidade angular do tipo  \textit{FeedForward}/\textit{Backward}(PID), que deve atuar no par de motores dos robôs de  forma a controlar suas velocidades bem como reduzir as assimetrias entre  os motores direito e esquerdo.\\

Usando a interface \textit{Bluetooth} presente no microcontrolador, foi criado um  protocolo simples de comunicação para realizar operações de telemetria e telecomando nos robôs. O \textit{Firmware} opera utilizando os dois núcleos  executando tarefas distintas: um núcleo é responsável pela comunicação  com o servidor, por meio do \textit{Bluetooth} e do protocolo criado, bem como  por tratar as interrupções geradas pelos \textit{Encoders} rotativos magnéticos;  já o outro núcleo executa a rotina de controle, com um período de amostragem de pelo menos $5$ms, para operar em uma frequência de pelo menos duas vezes mais rápido que o sistema que está rodando no \emph{host}, que como já mencionado trabalha até à $100$fps ($10$ms).\\

% resolução do encoder: 3 pulsos por revolução ou
% 6 pulsos se contar os dois canais ou
% 12 bordas de subida e descida nos dois canais por revolução
Para medir a velocidade de rotação, foram utilizados \textit{Encoders} magnéticos  rotativos, com resolução de $3$ pulsos por canal por revolução do eixo do motor. Para diminuir a  incerteza na estimativa da velocidade, é aplicado um filtro de \textit{Kalman},  considerando a planta (motores) como um sistema de primeira ordem. Os  parâmetros da planta (constante de tempo, zona morta e ganho PWM x  Velocidade angular) são estimados/calculados por mínimos quadrados na  rotina de calibração presente no \textit{Firmware}, sendo em seguida utilizados  para determinar os parâmetros do controlador.

% Organização do trabalho
% Este trabalhado é organizado da seguinte maneira: O Capítulo 1 corresponde à introdução do trabalho, por meio da qual é feita uma contextualização do problema e a proposta de solução para o mesmo; o capítulo 2 traz a fundamentação teórica utilizada para o desenvolvimento deste trabalho; no capítulo 3, os objetivos gerais e específicos são abordados; no capítulo 4 são apresentados os materiais e métodos utilizados para implementação do sistema; o capítulo 5 contém os resultados e discussões em que podem ser observadas as telas do sistema proposto; no capítulo 6 as conclusões deste trabalho são levadas em consideração, assim como sugestões de trabalhos futuros.