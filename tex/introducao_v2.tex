\section{INTRODUÇÃO}

% Start 01:
% Começar falando sobre controladores voltados a motores, controle de velocidade angular e controle de posição. A partir disso, mencionar os sensores mais utilizados e a importância/limitações com relação a sua precisão e como isso influência o desempenho dos controladores.

% Start 02:
% Começar falando sobre motores elétricos, sua importância em diversas áreas. mencionar a alta relevância em robótica móvel. introduzir sobre os controladores voltados para motores: controle de velocidade angular e controle de posição. A partir da introdução sobre os controladores falar da importância dos sensores e citar alguns tipos. Falar da relação/importância entre precisão/acurácia  dos sensores e a relação entre custo e limitações com relação ao tamanho do sensor para certas aplicações.

% Introdução ao trabalho desenvolvido:

%O trabalho apresenta uma proposta de sistema simples e eficiente de observação e controle de velocidade angular para motores elétricos de corrente contínua. Para a validação, o sistema foi implementado e embarcado no hardware de robôs com acionamento diferencial. 

% O trabalho propõe uma sistema de observação de velocidade angular e controle de velocidade angular para motores elétricos utilizando-se de encoders com baixa resolução. A metodologia foi analisada implementando-se o sistema de controle bem como a metologia 

% Organização do trabalho
% Este trabalhado é organizado da seguinte maneira: O Capítulo 1 corresponde à introdução do trabalho, por meio da qual é feita uma contextualização do problema e a proposta de solução para o mesmo; o capítulo 2 traz a fundamentação teórica utilizada para o desenvolvimento deste trabalho; no capítulo 3, os objetivos gerais e específicos são abordados; no capítulo 4 são apresentados os materiais e métodos utilizados para implementação do sistema; o capítulo 5 contém os resultados e discussões em que podem ser observadas as telas do sistema proposto; no capítulo 6 as conclusões deste trabalho são levadas em consideração, assim como sugestões de trabalhos futuros.

% Introdução ao projeto de Futebol de robôs, dando uma ideia geral sobre os desafios da categoria e uma visão geral sobre o sistema da equipe Poti