\section{INTRODUÇÃO}

% Notas:
% Separar a intro. em subseções: 
% Talvez: contextualização e intro. à robocup/VSSS
% Falar sobre a organização do trabalho

% Usei % [A Simple Speed Feedback System for Low Speed DC Motor Control in Robotic Applications] como ref. para o começo da intro.
Uma forma muito comum de acionamento de motores de corrente continua (CC) é por meio da modulação por largura de pulso, ou do inglês Pulse Width Modulation (PWM), que consiste basicamente em manipular a tensão média de alimentação do motor por meio de sinais digitais [?]. É comum em certas aplicações o uso do sinal \emph{PWM} para manipular a velocidade do motor, ou seja, um controle de malha aberta [?], esse tipo de sistema apenas regula o quão rápido o motor irá girar. Porém na maioria das aplicações o controle em malha aberta não é suficiente/satisfatório, como por exemplo, não há garantias que se aplicado tensões iguais (\emph{PWM}) em ambos os motores de um robô com acionamento diferencial (robôs com duas rodas tracionadas e independentes) o movimento será estritamente linear, isso devi-se as diferenças entre os dois conjuntos de motores/rodas que fazem com que esses não girem a uma mesma velocidade, mesmo sobre uma mesma tensão.\\

Em contraste com o sistema de malha aberta, um sistema de malha fechada utiliza um sinal da saída da planta como \emph{feedback}. Um sistema de controle de \emph{feedback} tende a manter uma relação de uma variável do sistema para outra por comparação entre as funções dessas variáveis e usando a diferença como meio de controle [?]. \\

% [Speed Measurement Algorithms for Low-Resolution Incremental Encoder Equipped Drives: a Comparative Analysis]
Portando, ter-se um bom \emph{feedback} em sistemas de controle em malha fechada é de extrema importância [?], tanto para se trabalhar em alta performance, quanto para garantir a estabilidade do sistema [?]. No contexto de controle de rotação de motores, os sensores mais comuns são os codificadores de eixo (\emph{Encoders}). Os dois principais tipos são: \emph{Encoders} absolutos e os incrementais. O primeiro fornece a informação da posição angular absoluta e o segundo uma informação da variação de giro. \\

A resolução desses sensores influencia fortemente no desempenho e até mesmo na viabilidade do controlador, porém, sensores com resoluções maiores são, por consequência, mais caros, fazendo-se que seja necessário um \emph{trade-off} entre custo e desempenho. Existem diversas técnicas que tentam melhorar a estimativa da velocidade sem que seja necessário o uso de sensores mais precisos [?][?][?], um tipo de técnica que geralmente é usada para esse fim são os estimadores de estado [?][?], que calculam uma estimativa da velocidade a partir dos dados do sensor e de algum conhecimento sobre o comportamento do sistema. Porém, também existe um \emph{trade-off} com relação a qual estimador usar. Bons estimadores exigem mais poder computacional e/ou exigem que tenha-se mais conhecimento sobre a planta. O que pode ser um problema para certos tipos de aplicações, como por exemplo, uma aplicação que deve rodar em um microcontrolador ($\mu$C) de um robô.\\



Este trabalho propõe um sistema de controle de velocidade de rotação para motores de corrente contínua do tipo \emph{Feedforward}/\emph{backward} e que utiliza o filtro de \emph{Kalman} como estimador de velocidade (a partir de \emph{Encoders} incrementais). A aplicação alvo deste trabalho foram robôs com acionamento diferencial com \emph{Encoders} de baixa resolução. Para a validação do trabalho foi feito a implementação do sistema proposto e embarcado em robôs com acionamento diferencial que foram projetados e desenvolvidos durante o trabalho. Os robôs projetados são voltados para a participação na Robocup[?] na categoria \emph{IEEE Very Small Size Soccer} (VSSS) por meio da equipe de Futebol de robôs do laboratório de robótica do Departamento de Computação e Automação (DCA) da UFRN, portanto, há limitações/desafios em seu projeto, principalmente com relação as suas dimensões, fazendo com que, por exemplo, seja difícil a utilização de \emph{Encoders} com alta precisão.\\


% Intro à Robocup, IEEE VSSS e Equipe Poti

A Equipe Poti de Futebol de Robôs surgiu no Departamento de Engenharia de Computação e Automação da Universidade Federal do Rio Grande do Norte (DCA-UFRN), sendo participante na categoria IEEE Very Small Size Soccer.\\

Nesse tipo de categoria três robôs de cada time são utilizados de forma cooperativa com o objetivo de fazer o gol. Cada robô possui uma função no jogo, como: o goleiro, o atacante e o defensor, de forma que dependendo da estratégia de jogo eles podem se alternarem para melhor desempenho do time. A arquitetura do sistema do Futebol de Robôs do Time Poti é formada por vários módulos, sendo eles: visão, localização, estratégia, controle e transmissão.\\

O módulo de visão consiste de uma câmera que fotografa o campo com os jogadores e a bola, a taxa de captura da câmera é o fator que limita o tempo de amostragem de todo o sistema, atualmente a equipe utiliza uma taxa de amostragem de $100$ quadros/\emph{frames} por segundo(FPS). O módulo de localização fornece a posição da bola e dos jogadores através das cores que foram predefinidas. O módulo de estratégia é responsável pelo próximo passe que os robôs deverão executar, a partir das imagens atuais. O módulo de controle é responsável por converte as referências de posição e orientação, definidas pela estratégia, em sinais que correspondem as velocidades que devem ser aplicadas nas rodas direita e esquerda de cada robô e o módulo de transmissão envia esse sinal para os robôs.\\

O \textit{Hardware} projetado para esses robôs conta com um microcontrolador dual core de 32\emph{bits} da empresa \emph{Espressif Systems} (ESP32) e \textit{Encoders} magnéticos de $3$ pulsos por revolução por canal. Foi embarcado  nesse microcontrolador um controle de velocidade angular do tipo  \textit{FeedForward}/\textit{Backward}(PID), que deve atuar no par de motores dos robôs de  forma a controlar suas velocidades bem como reduzir as assimetrias entre  os motores direito e esquerdo.\\

Usando a interface \textit{Bluetooth} presente no microcontrolador, foi criado um  protocolo simples de comunicação para realizar operações de telemetria e telecomando nos robôs. O \textit{Firmware} opera utilizando os dois núcleos  executando tarefas distintas: um núcleo é responsável pela comunicação  com o servidor, por meio do \textit{Bluetooth} e do protocolo criado, bem como  por tratar as interrupções geradas pelos \textit{Encoders} rotativos magnéticos;  já o outro núcleo executa a rotina de controle, com um período de amostragem de pelo menos $5$ms, para operar em uma frequência de pelo menos duas vezes mais rápido que o sistema que está rodando no \emph{host}, que como já mencionado trabalha até à $100$fps ($10$ms).\\

Para medir a velocidade de rotação, foram utilizados \textit{Encoders} magnéticos  rotativos, com resolução de $3$ pulsos por canal por revolução do eixo do motor. Para diminuir a  incerteza na estimativa da velocidade, é aplicado um filtro de \textit{Kalman},  considerando a planta (motores) como um sistema de primeira ordem. Os  parâmetros da planta (constante de tempo, zona morta e ganho \emph{PWM} x  Velocidade angular) são estimados/calculados por mínimos quadrados na  rotina de calibração presente no \textit{Firmware}, sendo em seguida utilizados  para determinar os parâmetros do controlador.


% Organização do trabalho
% Este trabalhado é organizado da seguinte maneira: O Capítulo 1 corresponde à introdução do trabalho, por meio da qual é feita uma contextualização do problema e a proposta de solução para o mesmo; o capítulo 2 traz a fundamentação teórica utilizada para o desenvolvimento deste trabalho; no capítulo 3, os objetivos gerais e específicos são abordados; no capítulo 4 são apresentados os materiais e métodos utilizados para implementação do sistema; o capítulo 5 contém os resultados e discussões em que podem ser observadas as telas do sistema proposto; no capítulo 6 as conclusões deste trabalho são levadas em consideração, assim como sugestões de trabalhos futuros.


% [High-Performance Speed Control of Electric Machine Using Low-Precision Shaft Encoder]

% SPEED CONTROL through accurate speed detection is
% an essential requirement in the field of high-performance
% motion control like machine tool applications. An ordinary
% speed control uses average speed information as a speed
% feedback signal that is obtained by counting the increased
% number of encoder pulses during the detection time or the
% sampling interval [1]. However, the average speed information
% is not good enough to control the motor speed precisely
% because average detection lag is produced and the speed
% information is lost during the sampling interval, especially in
% a low-speed range when a low-precision shaft encoder is used.
% Therefore, instantaneous speed observers based on distur-
% bance torque observer have been reported as a better method
% to avoid unstable system which can be caused by making
% sampling time longer to obtain more encoder pulses [2]–[4].
% However, the observers do not have good noise-rejection prop-
% erties in noisy environment in comparison with the Kalman
% filter known as an optimal state estimator [5]. The Kalman
% filter is a stochastic estimator that possesses the smoothing
% properties and the noise-rejection capability robust to the
% process and measurement noise. Even though the Kalman filter


% [Low Speed Control of Permanent Magnet Synchronous Motor Based on Instantaneous Speed Estimation]

% Abstract - Incremental-type encoder is widely used for speed
% detection in servo system. However, the speed controller of the
% system usually becomes unstable at low speed range using
% average speed estimation method. An instantaneous speed
% estimation scheme for low speed control of permanent magnet
% synchronous motor is proposed. The speed estimation adopts a
% full order state observer to calculate feedback speed. The method
% can achieve good response because it is possible to detect the
% speed exactly without detection dead time. The observer design
% considerations for application are analyzed. Furthermore, inertia
% identification based on recursive extended least square algorithm
% is presented to reduce sensitivity of the speed estimation.
% Experimental results show that low speed control performance
% using the instantaneous speed estimation with inertia
% identification is superior to that of conventional one.

% [...]

% Presently, incremental-type encoder is the most
% typical speed and position sensor for servo system considering
% the cost and the performance. Speed detection using the
% encoder is calculated by counting the number of pulses
% generated by the encoder and interval between them [1], [2].
% This method detects the averaging speed, and therefore, it
% causes a detection dead time. Under low speed conditions,
% especially, where the interval between the pulses is wider than
% a speed control period, the detection dead time usually makes
% the speed controller unstable.

% To improve the performance of the speed control system
% which has been degraded by the detection dead time, several
% instantaneous speed observers have been proposed by using
% state observation theory. Instantaneous speed estimation using
% reduced-order disturbance torque observer has simple
% structure and easy to implement [3], [4]. But the gains of
% observer and controller are limited to increase in real
% application for mechanical noise and oscillation of system.
% Recently, a full order state estimation algorithm has been
% proposed which is based on the optimal state estimation theory
% like Kalman filter [5], [6]. But this scheme is complicated to
% implement since it needs large calculation.


% [Speed Measumment Using Rotary Encoders for High Performance ac Drive]
% Abstd - Most of high performance speed control of electrical
% motors, like ac Vector Conhol, use mtay encoders. The accuracy
% obtaining the mtor speed fmm encoder signals is an essential
% mquimment to achieve a good dynamic msponse, but it usually
% implies a delay in the speed conbol loop, which can cause stability
% problems.